\documentclass{article}
\usepackage{amsmath}
\usepackage{amsfonts}
\usepackage{graphicx}
\usepackage{url}
\usepackage{float}
\usepackage[slovene]{babel}

\author{Lucija Fekonja}
\title{Domača naloga 1, Algoritmi}


\begin{document}
\selectlanguage{slovene}
    \maketitle

    \section*{Naloga 1: Turingov stroj}
    Turingov stroj je definiran s sedmerko $(\Sigma, \Gamma, q, F, \delta, B, Q)$. V našem primeru je
    \begin{itemize}
        \item $\Sigma = \{0, 1 \}$ je množica znakov, ki se lahko pojavijo na vhodnem traku,
        \item $\Gamma = \{0, 1, \_ \}$ je abeceda,
        \item $q = q0$ je začetno stanje
        \item $F = \{ sprejmi \}$ je množica končnih stanj, ki jih sprejme,
        \item $\delta$ funkcija je definirana v \path{turing_machine.txt} datoteki,
        \item $B = \_ $ je simbol, ki označuje prazen prostor na traku,
        \item $Q = $ \{ q0, samo0, preveri, preveriB, konecPreverjanja, zacetek, desnoX, desno1, desno0, vpisi, nazaj, vpisi0, nazaj0, naZacetek, brisi, spremeniVEna, sprejmi \} je množica stanj.
    \end{itemize}


    \section*{Naloga 2: Prevedba}
    Najprej dokažimo, da je $4$-SAT NP-poln, pri čemer izhajamo iz že znane ugotovitve, da je $3$-SAT NP-poln. To bomo izvedli v dveh korakih: najprej pokažemo, da je $4$-SAT NP problem, nato pa še, da je NP-težek.

    Nedeterministično polinomski (NP) odločitveni problem je tisti, pri katerem lahko v polinomskem času preverimo rešitev, vendar je ne moremo najti v polinomskem času. Za problem $4$-SAT je enostavno preveriti, ali dana konjunktivna normalna oblika (CNF) vrne TRUE. To lahko storimo tako, da za vsak člen konjunkcije preverimo, ali se vrednost izraza evalvira v TRUE. Tako preverjanje deluje v polinomskem času, kar kaže, da je $4$-SAT NP problem.

    Dokažimo, da je $4$-SAT NP-težek. Vemo, da je $3$-SAT NP-težek. V problemu $3$-SAT imamo konjunkcijo disjunkcij treh členov. Vsak člen konjunkcije je oblike $(a \vee b \vee c)$, kjer so $a, b$ in $c$ logične spremenljivke. Lahko pa to obliko tudi razpišemo na naslednji način:
    \[ (a \vee b \vee c) = (a \vee b \vee c \vee d) \wedge (a \vee b \vee c \vee \neg d) \text{.} \]
    Preverimo, da sta leva in desna stran enaki.

    \begin{itemize}
        \item Če je leva stran TRUE, mora biti vsaj ena od spremenljivk $a, b$ ali $c$ TRUE. To pomeni, da se tudi oba člena konjunkcije na desni evalvirata v TRUE. 
        \item Če je leva stran FALSE, so vse tri spremenljivke $a, b$ in $c$ FALSE. Da bi bila desna stran TRUE, bi morala biti tako $d$ kot tudi $\neg d$ TRUE, kar ni mogoče.
        \item Če je desna stran FALSE, so vse tri spremenljivke $a, b$ in $c$ FALSE, zato je tudi leva stran FALSE.
        \item Če je desna stran TRUE, mora biti vsaj ena od $a, b$ oziroma $c$ TRUE. Če bi bile vse tri spremenljivke FALSE, bi bil en od členov $(a \vee b \vee c \vee d)$ oziroma $(a \vee b \vee c \vee \neg d)$ FALSE, saj ne more hkrati veljati $d$ in $\neg d$.
    \end{itemize}
    Opazimo, da smo $3$-SAT pretvorili v $4$-SAT. Pretvorba ima polinomsko časovno zahtevnost, zato je tudi $4$-SAT NP-težek problem.
    
    Na enak način lahko dokažemo, da je $5$-SAT NP-poln. V $5$-SAT imamo konjunkcijo disjunkcij petih členov. Za nedeterministično izbran vhod $\vec{x}$ lahko v polinomskem času preverimo, ali konjunktivna normalna oblika (CNF) vrne TRUE, zato je $5$-SAT NP problem.
    Nato prevedemo $4$-SAT v $5$-SAT. Vsak člen v $4$-SAT lahko razčlenimo na podoben način kot prej:
    \[ (a \vee b \vee c \vee d) = (a \vee b \vee c \vee d \vee e) \wedge (a \vee b \vee c \vee d \vee \neg e) \text{.} \]
    Kot prej preverimo, da leva in desna stran vedno vrneta isto vrednost:
    \begin{itemize}
        \item Če je leva stran TRUE, mora biti vsaj ena od spremenljivk $a, b$, $c$ ali $d$ TRUE. To pomeni, da se tudi oba člena konjunkcije na desni evalvirata v TRUE. 
        \item Če je leva stran FALSE, so vse štiri spremenljivke $a, b$, $c$ in $d$ FALSE. Da bi bila desna stran TRUE, bi morala biti tako $e$ kot tudi $\neg e$ TRUE, kar ni mogoče.
        \item Če je desna stran FALSE, so vse štiri spremenljivke $a, b$, $c$ in $d$ FALSE, zato je tudi leva stran FALSE.
        \item Če je desna stran TRUE, mora biti vsaj ena od $a, b$, $c$ ali $d$ TRUE. Če bi bile vse štiri spremenljivke FALSE, bi bil en od členov $(a \vee b \vee c \vee d \vee e)$ oziroma $(a \vee b \vee c \vee d \vee \neg e)$ FALSE, saj ne more hkrati veljati $e$ in $\neg e$.
    \end{itemize}
    
    Poleg tega transformacija iz $4$-SAT v $5$-SAT zahteva polinomski čas. Tako je $5$-SAT NP in NP-težek, kar pomeni, da ga lahko štejemo kot NP-poln problem.
    % Za $4$-SAT ne obstaja algoritem, ki bi naštel rešitev v polinomskem času, zato ga štejemo kot NP problem in ne P.


    \section*{Naloga 3: Rodovne funkcije}
    \subsection*{Rodovna funkcija za izbiro avta}

    Rodovna funkcija za vsak del avta:
    \begin{itemize}
        \item \textit{Motor:} $A(x) = 4x \cdot 2x = 8x^2$, kjer $4x$ predstavlja $4$ vrste motorja, $2x$ pa dve vrsti pogona. 
        Ker moramo izbrati vrsto motorja kot tudi vrsto pogona, faktorja $4x$ in $2x$ med sabo pomnožimo.
        \item \textit{Zunanjost:} $B(x) = 6x$, saj lahko izberemo $6$ različnih barv.
        \item \textit{Notranjost:} $C(x) = 2x + 2x^2$, kjer člen $2x = x + x$ predstavlja izbiro ustnjenih sedežev ali športnih.
        Člen $2x^2 = x^2 + x^2$ pa, da smo izbrali obojne ustnjene in športne sedeže. Izbiri sta dve, saj lahko izberemo sprednje
        sedeže športne in zadnje ustnjene ali pa obratno.
        \item \textit{Dodatna oprema:} $D(x) = 1 + 3x + \binom{3}{2} x^2 + x^3 = 1 + 3x + 3 x^2 + x^3 $, saj lahko ne izberemo nobene dodatne opreme, odtod člen $1$,
        lahko izberemo eno od treh možnosti dodatne opreme, odtod člen $3x$, lahko izberemo dve možnosti, dobimo člen $3x^2$,
        ali pa izberemo vse tri dodatke, dobimo člen $x^3$.
    \end{itemize}

    \noindent
    Rodovna funkcija, ki nam našteje vse možne kombinacije, je produkt zgornjih, saj izbiramo motor,
    zunanjost, notranjos in dodatno opremo. Dobimo
    \begin{align*}
        f(x) &= A(x) \cdot B(x) \cdot C(x) \cdot D(x) \\
        &= 8 x^2 \cdot 6x \cdot (2x + 2x^2) \cdot (1 + 3x + 3x^2 + x^3) \\
        &= 96 x^4 + 96 x^5 + 288 x^5 + 288 x^6 + 288 x^6 + 288 x^7 + 96 x^7 + 96 x^8 \\
        &= 96 x^4 + 384 x^5 + 576 x^6 + 384 x^7 + 96 x^8
    \end{align*}

    \subsection*{Fibonaccijeva števila drugega reda}

    $n$-to Fibonaccijevo število drugega reda je definirano kot
    $$ \mathfrak{F}_n = 
    \begin{cases}
        0, & n = 0 \\
        1, & n = 1 \\
        \mathfrak{F}_{n-1} + \mathfrak{F}_{n-2} + F_n, & n > 1
    \end{cases}
    \text{.} $$
    Rodovna funkcija zanje je $\mathfrak{G} (x) = \sum_{n=0}^{\infty} \mathfrak{F}_n x^n $.
    Izrazimo $\mathfrak{G} (x), x \mathfrak{G} (x)$ in $x^2 \mathfrak{G} (x)$ s koeficienti Fibonaccijevih 
    števil in Fibonaccijevih števil druge vrste:
    \begin{align*}
        \mathfrak{G} (x) &= \mathfrak{F}_0 + \mathfrak{F}_1 x + \mathfrak{F}_2 x^2 + \cdots + \mathfrak{F}_k x^k + \cdots \\
        &= \mathfrak{F}_0 + \mathfrak{F}_1 x + (\mathfrak{F}_1 + \mathfrak{F}_0 + F_2) x^2 + \cdots + (\mathfrak{F}_{k-1} + \mathfrak{F}_{k-2} + F_k) x^k \\
        x \mathfrak{G} (x) &= \mathfrak{F}_0 x + \mathfrak{F}_1 x^2 + \mathfrak{F}_2 x^3 + \cdots + \mathfrak{F}_{k-1} x^k + \cdots \\
        &= \mathfrak{F}_0 x + \mathfrak{F}_1 x^2 + (\mathfrak{F}_1 + \mathfrak{F}_0 + F_2) x^3 + \cdots + (\mathfrak{F}_{k-2} + \mathfrak{F}_{k-3} + F_{k-1}) x^k \\
        x \mathfrak{G} (x) &= \mathfrak{F}_0 x^2 + \mathfrak{F}_1 x^3 + \mathfrak{F}_2 x^4 + \cdots + \mathfrak{F}_{k-2} x^k + \cdots \\
        &= \mathfrak{F}_0 x^2 + \mathfrak{F}_1 x^3 + (\mathfrak{F}_1 + \mathfrak{F}_0 + F_2) x^4 + \cdots + (\mathfrak{F}_{k-3} + \mathfrak{F}_{k-4} + F_{k-2}) x^k \\
    \end{align*}
    Vstavimo v naslednji izraz in ga poenostavimo:
    \begin{align*}
        (1 - x - x^2) \mathfrak{G} (x) &= \mathfrak{F}_0 + (\mathfrak{F}_1 - \mathfrak{F}_0) x + (\mathfrak{F}_1 + \mathfrak{F}_0 + F_2 -\mathfrak{F}_1 - \mathfrak{F}_0) x^2 + \\
        &\quad + (\mathfrak{F}_2 + \mathfrak{F}_1 + F_3 - \mathfrak{F}_1 - \mathfrak{F}_0 - F_2 - \mathfrak{F}_1) x^3 + \\
        &\quad + (\mathfrak{F}_3 + \mathfrak{F}_2 + F_4 - \mathfrak{F}_2 - \mathfrak{F}_1 - F_3 - \mathfrak{F}_1 - \mathfrak{F}_0 - F_2) x^4 + \cdots + \\
        &\quad + (\mathfrak{F}_{k-1} + \mathfrak{F}_{k-2} + F_k - \mathfrak{F}_{k-2} - \mathfrak{F}_{k-3} - F_{k-1} - \mathfrak{F}_{k-3} - \mathfrak{F}_{k-4} - F_{k-2}) x^k
    \end{align*}
    Sedaj uporabimo definicijo Fibonaccijevih števil drugega reda. Natančneje, uporabimo 
    \begin{align*}
        \mathfrak{F}_0 &= 0, \\
        \mathfrak{F}_1 &= 1, \\
        \mathfrak{F}_1 - \mathfrak{F}_{n-1} - \mathfrak{F}_{n-2} - F_n &= 0.
    \end{align*}
    Dobimo
    \begin{align*}
        (1 - x - x^2) \mathfrak{G} (x) &= \mathfrak{F}_0 + (\mathfrak{F}_1 - \mathfrak{F}_0) x + \sum_{k=2}^{\infty} F_k x^k \\
        &= x + \sum_{k=2}^{\infty} F_k x^k \\
        &= x + \sum_{k=0}^{\infty} F_k x^k - F_0 - F_1 x \\
        &= x + \sum_{k=0}^{\infty} F_k x^k - x \\
        &= \sum_{k=0}^{\infty} F_k x^k
    \end{align*}
    Označimo s $F(x) = \sum_{k=0}^{\infty} F_k x^k$ rodovno funkcijo za Fibonaccijeva števila. Potem je 
    $$  \mathfrak{G} (x) = \frac{1}{1 - x - x^2} F(x) \text{.}$$
    Rodovno funkcijo Fibonaccijevih števil lahko zapišemo tudi s $F(x) = \frac{x}{1 - x - x^2}$, zato 
    imamo posplošitev 
    $$  \mathfrak{G} (x) = \frac{1}{x} F(x)^2 \text{.}$$
    Vemo, da je koeficient vrste $F(x)^2$ pri členu z $x^k$ enak $\frac{(n-1) F_n + 2n F_{n-1}}{5}$,
    torej je koeficient vrste $\frac{1}{x} F(x)^2$ pri členu z $x^k$ enak $$ \mathfrak{F}_n = \frac{n F_{n+1} + 2 (n+1) F_n}{5} \text{.}$$


\end{document}
\documentclass[fleqn]{article}
\usepackage{amsmath}
\usepackage{amsfonts}
\usepackage{graphicx}
\usepackage{url}
\usepackage{xcolor}
\usepackage{listings}
\usepackage{subcaption}
\usepackage[slovene]{babel}
\usepackage{algorithm}
\usepackage{algpseudocode}

\DeclareMathOperator{\lcm}{lcm}

\title{Tretja domača naloga pri algoritmih}
\author{Lucija Fekonja}

\begin{document}

\maketitle

\section*{Naloga 1: Najmanjši skupni množitelj}
Najmanjši skupni množitelj števil $a$ in $b$ lahko izračunamo s pomočjo največjega skupnega delitelja po formuli 
\begin{equation}
    \label{lcm}
    \centering
    \lcm (a, b) = \frac{| a \cdot b |}{\gcd(a, b)} \text{.}
\end{equation}

Naša naloga je, da s pomočjo operatorja $\gcd$ najdemo najmanjši skupni večkratnih števil $a_1 \dots a_n \in \mathbb{Z}$. To lahko naredimo iterativno v eni zanki.
Najprej shranimo prvi dve števili $a_1$ in $a_2$ v novi spremenljivki $x$ in $y$. Nato izračunamo najmanjši skupni večkratnik $x$ in $y$ po formuli \eqref{lcm} in ga shranimo v novo spremenljivko $z$.
Prej definiran $x$ nastavimo na $z$, spremenljivko $y$ pa na naslednje število $a_i$. Ponavljamo dokler ne pridemo do $a_n$. Pseudokoda je naslednja (v njej začnemo indeksirati z $1$):
\begin{algorithm}
    \caption{Algoritem za iskanje najmanjšega skupnega večkratnika števil v seznamu $A = [a_1 \dots a_n]\text{.}$}
    \begin{algorithmic}
        \Procedure{lcm}{$A$}
            \State $x \leftarrow A[1]$
            \State $x \leftarrow A[2]$
            \State \textbf{for} $i = 3 \dots n$ \textbf{do}
            \State \qquad $z \leftarrow \frac{| x \cdot y |}{\gcd (x, y)}$  
            \State \qquad $x \leftarrow z$
            \State \qquad $y \leftarrow A[i]$
            \State \textbf{end for}
            \State \textbf{return} $x$
        \EndProcedure
        \end{algorithmic}    
\end{algorithm}

Algoritem v vsaki iteraciji izračuna najmanjši skupni večkratnik števila $a_i$ in vseh števil $a_1 \dots a_{i-1}$, ki jih je že obhodil.
Zato je izhod ravno njihov skupni najmanjši večkratnik.

\pagebreak

\section*{Naloga 2: Modularno potenciranje}

Naša naloga je napisati algoritem, ki izračuna $a^b \mod n$, tako da obhodi bite števila $b$ iz desne proti levi.
Pri tem imamo $b$ shranjen v dvojiškem zapisu.
To lahko storimo rekurzivno s pomočjo formule, ki velja tako v $\mathbb{Z}_n$ kot tudi v $\mathbb{Z}$:

\[
   a^b = \begin{cases}
    1 \mod n & b = 0 \\
    a^{b/2} \mod n & \text{$b>0$ in $b$ je sod} \\
    a^{b-1} \mod n & \text{$b>0$ in $b$ je lih} \\
   \end{cases} \text{.}
\]

Oglejmo si kaj nam pove formula.
\begin{enumerate}
    \item Ustavitveni pogoj $b = 0$. Tedaj vrnemo $1$. 
    \item Če je $b$ sodo število, torej če je njegov najbolj desni bit enak $0$, moramo izračunati $a^{b/2} \mod n$. V dvojiškem zapisu $b/2$ izračunamo tako, da preprosto izbrišemo najbolj desno ničlo. Vidimo, da lahko začetni $a^b$ dobimo po $a^b = a^{b/2} \cdot a^{b/2} \mod n$.
    \item Če pa je $b$ liho število, torej če je njegov najbolj desni bit enak $1$, pa izračunamo  $a^{b-1} \mod n$ in začetni $a^b$ dobimo po enačbi $a^b = a \cdot a^{b-1} \mod n$. Pri tem lahko $b-1$ v dvojiškem zapisu izračunamo tako, da najbolj desni bit iz $1$ spremenimo v $2$.

\end{enumerate}

Ker moramo v vsakem klicu preveriti zgolj najbolj desni bit števila $b$, res obhodimo $b$ od desne proti levi.
Pseudokoda algoritma je zapisana spodaj.

\begin{algorithm}
    \caption{Modularno množenje $a^b \mod n$.}
    \begin{algorithmic}
        \Procedure{modularno množenje}{$a, b, n$}
            \State Zapišimo $b$ v bitih kot $b = b_n \dots b_0$
            \State \textbf{if} $b = 0$
            \State \qquad \textbf{return} $1$
            \State \textbf{else if} $b_0 = 0$  
            \State \qquad $b \leftarrow b_n \dots b_1$
            \State \qquad d = \Call{modularno množenje}{} $(a, b, n)$
            \State \qquad \textbf{return} $(d \cdot d) \mod n$
            \State \textbf{else} (* $b_0 = 1$ *)
            \State \qquad $b \leftarrow b_n \dots b_1 0$
            \State \qquad d = \Call{modularno množenje}{} $(a, b, n)$
            \State \qquad \textbf{return} $(a \cdot d) \mod n$
        \EndProcedure
        \end{algorithmic}    
\end{algorithm}

\pagebreak

\section*{Naloga 3: Problem trgovskega potnika}

Naj $H_n$ označuje pot pridobljeno z hevristiko najbližje točke, ko vstavimo $n$ točk, in naj $H_n^*$ označuje optimalno pot.
Naj bo $c(v, u)$ cena oz. razdalja med točkama $u$ in $v$. Podobno naj bo $c(H_i)$ vsota cen vseh povezav v poti $H_i$.

V vsakem koraku opisanega algoritma izbrižemo eno povezavo in dodamo dve novi. 
Recimo, da je v $H_{n-1}$ obstajala povezava $u \rightarrow w$ in da še $v$ ni v ciklu.
Naj bo $v$ točka, ki je najbližje ciklu in jo v naslednjem koraku vstavimo takoj za $u$.
Tako dobimo povezave $u \rightarrow v \rightarrow w$.
Posledično se skupna cena poviša za 
\begin{equation}
    \label{cena}
    c(u, v) + c(v, w) - c(u, w) \text{.}
\end{equation}
Po trikotniški neenakosti vemo, da je $c(u, v) \geq c(u, w) - c(v, w)$, zato lahko omejimo enačbo \eqref{cena} z
$$ c(u, v) + c(v, w) - c(u, w) \leq 2 c(u, v) \text{.}$$
Tako lahko ceno $n$-tega koraka omejimo s pomočjo cene $n-1$-vega koraka na naslednji način:
$$ c(H_n) \leq c(H_{n-1}) + 2 c (u, v) \text{.} $$
Torej če dodajamo povezave $(u_1, v_1) \dots (u_n, v_n)$, lahko ceno poti $H_n$ omejimo z
\begin{align*}
    c(H_n) \leq c(H_{n-1}) + 2 c (u_n, v_i) \\
    &\leq c(H_{n-1}) + 2 c (u_{n-1}, v_{n-1}) + 2 c(u_n, v_n) \\
    &\leq \cdots \\
    &\leq 2 \sum_{i=1}^{n} c(u_i, v_i)
\end{align*}

Sedaj si oglejmo kako deluje algoritem. Začne se s poljubno izbiro mesta $v$ in tvori cikel sam vase.
Nato na vsakem koraku poišče masto $u$, ki še ni v ciklu, ampak je najbližje kateremu koli mestu v ciklu, recimo $v$, in $u$ postavi takoj za $v$ v cikel.

Oglejmo si kako deluje Primov algoritem za iskanje minimalnega vpetega drevesa (ang. minimal spanning tree oz. MST).
Na začetku izbere naključno vozlišče v grafu $G$. Nato izbere povezavo, katere eno vozližče je že bilo obiskano, drugo pa ne in ima najmanjšo možno ceno. 
Ko najde to povezavo, jo doda vljučno z njenimi vozlišči v MST. Ponavlja, dokler niso vsa vozlišča iz $G$ v MST.

V principu sta oba algoritma enaka, saj na vsakem koraku iteracije dodamo vozlišče, ki je njmanj oddaljeno od katerega koli že dodanega vozlišča.

V Primovem algoritmu je cena vseh dodanih povezav enaka $\sum_{i=1}^{n} c(u_i, v_i)$. Torej 
$$ c(MST) = \sum_{i=1}^{n} c(u_i, v_i) \text{.} $$

Po definiciji je minimalno vpeto drevo tisto, katerega vsota teži vozlišč je najmanjša možna, kar implicira $c(MST) \leq c (H_n^*)$.

Tako pridemo do željene neenakosti:
$$ c (H_n) \leq 2 \sum_{j=1}^{n} c(u_i, v_i) = 2 c(MST) \leq c (H_n^*) \text{.} $$

To pa ravno dokazuje dvo aproksimacijo. Namreč slednje pove, da cena heuristike ni večja od dvakratnika cene optimalne poti.
Tako smo zaključili naš dokaz. 
\end{document}
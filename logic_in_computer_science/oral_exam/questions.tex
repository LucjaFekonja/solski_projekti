\documentclass[fleqn]{article}
\usepackage{amsmath}
\usepackage{xcolor}
\usepackage[slovene]{babel}

\title{Questions for oral exam, Logic in CS}
\author{Lucija Fekonja}

\begin{document}

\maketitle

\section{Logic and type theory}

\subsection{Propositional logic}
\textit{Lecture 1}
\begin{enumerate}
    \item Define a formula of propositional logic $\Phi$.
    \begin{itemize}
        \item Describe the rules.
    \end{itemize}
    \item How can we describe formulas?
    \item What is the grammar for propositional logic formula?
    \item Why are proof rules useful?
    \item Define tautologies.
    \item Write and explain proof rules for propositional logic
    \begin{itemize}
        \item Conjunction introduction rules
        \item Conjunction elimination rules
        \item Disjunction introduction rules
        \item Disjunction elimination rules
        \item Implication introduction rules
        \item Implication elimination rules
        \item How would you explain Implication rules?
        \item Negation introduction rules
        \item Negation elimination rules
        \item True rules
        \item False rules
    \end{itemize}
    \item How can we write proofs?
    \item EXAMPLE \-- Prove $p \wedge q \Rightarrow q \wedge p$ with a list.
    \item EXAMPLE \-- Prove $p \wedge q \Rightarrow q \wedge p$ with a tree.
    \item EXAMPLE \-- Prove $p \wedge (q \vee r) \Rightarrow (p \wedge q) \vee (p \wedge r)$ with a list.
    \item EXAMPLE \-- Prove $p \wedge (q \vee r) \Rightarrow (p \wedge q) \vee (p \wedge r)$ with a tree.
    \item How do we come to a contradiction?
    \item Write a rule with which we can apply proof by contradiction?
    \item EXAMPLE \-- Prove $\Phi \vee \neg \Phi$.
    \item Simplify grammar.
\end{enumerate}

\subsection{Simple types and their proof rules and computation rules}
\textit{Lecture 1}
\begin{enumerate}
    \item Give grammar of type theory with sum and product types.
    \begin{itemize}
        \item What is necessery for a type theory grammar to have? (3)
    \end{itemize}
    \item Explain the analogy between types and formulas.
    \begin{itemize}
        \item Product type introduction rules
        \item Product type elimination rules
        \item Analogy to what is product type?
        \item Function type introduction rules
        \item Function type elimination rules
        \item Analogy to what is function type?
        \item How would you explain function rules?
        \item Sum type introduction rules
        \item Sum type elimination rules
        \item Rule for empty type
    \end{itemize}
    \item Name this analogy.
    \item Give computation rules
    \begin{itemize}
        \item Product type
        \item Sum type 
        \item Function type
    \end{itemize}
    \item EXAMPLE -- Derive a term of type $\sigma \times \tau \rightarrow \tau \times \sigma$.
    \item EXAMPLE -- Write a derivation of distributiviry to derive the term $\sigma \times (\tau + \Phi) \rightarrow (\sigma \times \tau) + (\sigma \times \Phi)$.
\end{enumerate}

\subsection{The universes Prop and Type}
\textit{Lecture 2 professor's notes}
\begin{enumerate}
    \item With what can we replace the grammar for propositional logic? 
    \item Give rules in the Prop universe
    \begin{itemize}
        \item False
        \item Conjunction
        \item Disjunction
        \item Implication
    \end{itemize}
    \item How do we distinguish between judgements $\Phi : Prop$ and $\Phi$?
    \item With what can we replace the grammar for Types? 
    \item Give rules in the Type universe 
    \begin{itemize}
        \item Empty type
        \item Product
        \item Sum
        \item Function
    \end{itemize}
    \item How do we distinguish between judgements $A : Type$ and $t : A$?
\end{enumerate}

\subsection{Quantifiers}
\textit{Lecture 2}
\begin{enumerate}
    \item How do we also call predicate logic?
    \item How do we get predicate logic form propositional logic?
    \item How can we extend propositional logic? (3)
    \begin{itemize}
        \item aka. What do we add to the grammar?
    \end{itemize}
    \item Give well formed rules for quantifiers
    \begin{itemize}
        \item $\forall$
        \item $\exists$
    \end{itemize}
    \item Give proof rules for quantifiers
    \begin{itemize}
        \item $\forall i$
        \item $\forall e$
        \item $\exists i$
        \item $\exists e$
    \end{itemize}
    \item What is the standard approach to predicate logic?
    \begin{itemize}
        \item What do we choose?
        \item EXAMPLE -- Give examples.
        \item How do we use it?
        \item EXAMPLE -- Give examples.
        \item How do we use it's properties?
        \item EXAMPLE -- Give examples.
        \item What do predicates express?
    \end{itemize}
    \item What is an atomic formula?
    \item EXAMPLE -- Give a simple example of atomic formula.
    \item EXAMPLE -- Give an example of using predicate logic using terms for $t, \cdot, 0, 1, <$.
    \begin{itemize}
        \item What here are terms?
        \item What here are atomic formulas?
        \item What holds for variables?
        \item What are free variables?
        \item Does this formula have free variables?
        \item Where is this formula true and where is it false?
    \end{itemize} 
    \item What is the major limitation of traditional predicate logic?
    \item Where is the problem from the perspective of programming?
    \item How can we solve this problem?
    \item Write an unified grammar.
    \item With what can we replace grammar?
    \item What does type theory give us?
    \item What does predicate logic give us?
    \item How can we combine the two? 
\end{enumerate}


\subsection{Interesting types}
\textit{Lecture 3}
\begin{enumerate}
    \item How do we call "adding new types"?
    \item What are formation rules used for?
    \item EXAMPLE -- Give examples of formation rules.
    \item What do introduction rules give?
    \item EXAMPLE -- Give introduction rules for natural numbers. 
    \item What do elimination rules give?
    \item Give four examples of type constructors / new types.
    \item Formation rules for list type constructor.
    \item How else can we write formation rule?
    \item Introduction rules for list type constructor.
    \item How else can we write introduction rule?
    \item How do we construct a list?
    \item Formation rules for vector type constructor.
    \item How else can we write formation rule?
    \item Introduction rules for vector type constructor.
    \item How else can we write introduction rule?
    \item How is $A \rightarrow B$ defined?
    \item Formation rule for product type
    \item Introduction rule for product type
    \item Elimination rule for product type
    \item Computation rule for product type
    \item How is $A \times B$ defined?
    \item Formation rule for dependent sum type
    \item Introduction rule for dependent sum type
    \item Elimination rules for dependent sum type
    \item Computation rules for sum type
\end{enumerate}

\subsection{The BHK interpretation and the Curry-Howard correspondence}
\textit{Lecture 3}
\begin{enumerate}
    \item Which elements do the following types have?
    \begin{itemize}
        \item $0$
        \item $A \times B$
        \item $A + B$
        \item $A \rightarrow B$
        \item $\pi x: A.B$
        \item $\sum x: A.B$
    \end{itemize}
    \item What does BHK stand for?
    \item What did they do?
    \item Give proofs for the following terms:
    \begin{itemize}
        \item $\bot$
        \item $\Phi \wedge \Psi$
        \item $\Phi \vee \Psi$
        \item $\Phi \rightarrow \Psi$
        \item $\forall x : A. \Phi$
        \item $\exists x: A. \Phi$
    \end{itemize}
    \item What does Curry-Howard correspondance say?
    \item For what kind of theory does it hold?
    \item What can we use instead of:
    \begin{itemize}
        \item $\bot$
        \item $\Phi \wedge \Psi$
        \item $\Phi \vee \Psi$
        \item $\Phi \rightarrow \Psi$
        \item $\forall x : A. \Phi$
        \item $\exists x: A. \Phi$
    \end{itemize}
    \item Who developed this approach?
    \item When?
    \item What approach does Lean take?
    \item Describe this approach.
    \item Define subuniverse
    \item What is a subterminal property?
    \item What do we also need here?
    \item How do we change the formation rule for product type?
    \item What is Scott-Prawitz interpretation of logic?
    \item How did they define the following terms?
    \begin{itemize}
        \item $\bot$
        \item Which equality do we have using this definition?
        \item $\Phi \wedge \Psi$
        \item $\Phi \vee \Psi$
        \item $\Phi \rightarrow \Psi$
        \item $\forall x : A. \Phi$
        \item $\exists x: A. \Phi$
    \end{itemize}
    \item Explain $\Phi \wedge \Psi$ of Scott-Prawitz interpretation
    \begin{itemize}
        \item If we want to derive introduction rule
        \item If we want to derive elimination rule 
    \end{itemize}    
\end{enumerate}

\subsection{Natural numbers, lists and vectors as inductive types}
\textit{Lecture 4}
\textbf{Nat}
\begin{enumerate}
    \item Write out induction on natural numbers
    \item Define the eliminator for natural numbers
    \item What do we need do define for the eliminator? (name it)
    \item Give computation rules for natural numbers
    \begin{itemize}
        \item How do we use the recursor on 0?
        \item How do we use the recursor on succ n?   
    \end{itemize}
    \item EXAMPLE -- Proposition (fun), result, function
    \item How do we define a sum of two natural numbers?
    \item Give an improved version of n+m
    \item EXERCISE -- Derive $succ 0 + succ 0$
\end{enumerate}

\textbf{Lists}
\begin{enumerate}
    \item Give constructors for lists (2)
    \item Give eliminator for lists
    \item Give computation rules for lists (2)
    \item EXERCISE - Define append on lists using $R_{list}$.
\end{enumerate}

\textbf{Vectors}
\begin{enumerate}
    \item Give constructors for vectors (2)
    \item Give eliminator for vectors
    \item Give computation rules for vectors (2)
\end{enumerate}

\textbf{Propositional equality}
\begin{enumerate}
    \item Give formation rule
    \item Define constructur $rfl$
    \item Define eliminator
    \item Give computation rule
    \item What is the intuition for propositional equality?
\end{enumerate}

\textbf{Judgemental equality}
\begin{enumerate}
    \item What do we define?
    \item Give formulation rule for judgemental equality.
    \item Give substitution rule for judgemental equality.
\end{enumerate}


\subsection{Tantologies and satisfiability}
\textit{(Lecture 5)}
\begin{enumerate}
    \item Determine grammar of propositional logic.
    \item What is a formula?
    \item What are it's arguments?
    \item What can we determine with a formula?
    \item EXAMPLE - Formula in propositional Logic.
    \item Define a valid tantology.
    \item Define a satisfiable formula.
    \item What is such $b$ called?
    \item Define equivalence.
    \item How can we establish that a formula is a tantology? (2)
    \item In what time do they work?
    \item What theorems do we have for natural deduction?
    \item Give soundness theorem.
    \item Give completeness thorem.
    \item What is the size of the truth table.
    \item When is a formula a tantology based on the truth table?
    \item What can we also determine with a truth table?
    \item How?
\end{enumerate}

%%%%%%%%%%%%%%%%%%%%%%%%%%%%%%%%%%%%%%%%%%%%%%%%%%%%%%%%%%%%%%%%%%%%%%%%%%

\section{SAT problem}

\subsection{Normal forms for formulas}
\textit{Lecture 5}
\begin{enumerate}
    \item Define the negation normal form.
    \item Define the grammar of NNF.
    \item What does deMorgan say about NNF?
    \item How can we rewrite a formula for it to be written in NNF? (5) 
    \item In what time can we compute NNF from the original?
\end{enumerate}

\subsection{Conjunctive and  normal form}
\textit{Lecture 5}
\begin{enumerate}
    \item Define a literal
    \item Define a clause
    \item What is a clause of length 0?
    \item Define conjunctive normal form
    \item What is a CNF of length 0?
    \item Define a co-clause
    \item Define a disjunctive normal form.
    \item What proposition do we make on CNF and DNF?
    \item Give an recursive algorithm for computing $\Phi^{CNF}$. (6)
    \begin{itemize}
        \item What do we assume for $\Phi$?
    \end{itemize}
    \item In what time can we compute CNF and DNF?
    \item -------------???-------------- 
\end{enumerate}

\subsection{SAT problem}
\textit{Lecture 5}
\begin{enumerate}
    \item What is the input?
    \item What is the output?
    \item What is the related problem to SAT?
    \item What is it's input?
    \item What is it's output?
    \item How are this two problems related?
    \item In what time can $\Phi$ be solved?
    \begin{itemize}
        \item In what form does $\Phi$ have to be written for that to be the case?
    \end{itemize}
    \item Exactly when is a DNF satisfiable?
    \item What is the open question?
    \item What would be proved if this it true?
    \item When is SAT solving sufficient?
    \item What transformation do we define?
    \item What is the imageof this mapping?
    \item What property does this transformation have?
    \item In what time can it be computed?
    \item For what is SAT solving useful?
    \item What is a checkable solution called?
    \item EXAMPLE - Give an example of a problem that can be solved with SAT
    \begin{itemize}
        \item What is a 3-coloring?
        \item What is the question?
        \item How do we solve it?
        \item What do we define?
        \item How does the formula look like?
        \item What holds for every edge?
        \item When does a graph have a 3-colouring?
    \end{itemize}
    \item Explain DPLL algorithm
\end{enumerate}


\subsection{Examples}
\textit{Lecture 6}
\begin{enumerate}
    \item We change the algorithm for SAT solver for it to be informative. How?
    • Input?
    • Output?
    \item In what time can SAT work?
    \item EXAMPLE - $\Phi = \neg (( x \vee \neg y) \wedge \neg (x \wedge \neg (y \wedge z)))$
    \begin{itemize}
        \item Create a tree
        \item Substitute
        \item Create a list of clauses
    \end{itemize}
    \item EXAMPLE - Sudoku
    \begin{itemize}
        \item How do we represent a problem?
        \item How many variables do we have?
        \item What is the goal?
        \item What holds for square 11 because there has to be to least one number in every square?
        \item How many such clauses exist?
        \item What holds for square 11 because there has to be to most one number in every square?
        \item How many such clauses exist?
        \item How many clauses exist?
        \item How many clauses exist for a 4x4 sudoku?
    \end{itemize}
\end{enumerate}

%%%%%%%%%%%%%%%%%%%%%%%%%%%%%%%%%%%%%%%%%%%%%%%%%%%%%%%%%%%%%%%%%%%%%%%%%%%

\section{Predicate logic and bounded model checking}

\subsection{Signatures, terms and formulas}
\textit{Lecture 7}
\begin{enumerate}
    \item EXAMPLE - Give example properties that could be used in computer systems (4)
    \item Define syntax
    \item Define semantics
    \item EXAMPLE - Axioms
    \begin{itemize}
        \item Example axiom - squares
        \item Where is this true?
        \item Where is it false?
        \item Example axiom - inverse
        \item Where is this axiom true?
        \item Where is it false?
        \item What inthe past three examples is new vocabulary?
    \end{itemize}
    \item How do we specify vocabulary?
    \item Define signature
    \item What does every predicate have?
    \item What does each function have?
    \item What are $(P, F)$ in CS example?
    \item What are $(P, F)$ in axioms example?
    \item What does signature determine? (2)
    \item What do terms express?
    \item What do formulas express?
    \item What is always a term?
    \item How can we construct terms?
    \item EX - What are terms in CS example?
    \item EX - What are terms in axiom example?
    \item What are always formulas?
    \item How can we construct formulas from terms?
    \item How can we construct formulas from formulas?
    \item How else can we construct formulas?
    \item What here are atomic formulas?
    \item How are the semantics given?
\end{enumerate}

\subsection{Free and bound variables, sentences, substitution}
\begin{enumerate}
    \item Interpret $\exists y. x=y\times y$ in N
    \begin{itemize}
        \item What does the formula express?
        \item What is a free variable?
        \item What is a bound variable?
        \item In what scope?
    \end{itemize}
    \item What kind of variables are not in axiom exercise (exercise 2)?
    \item What is such a formula called?
\end{enumerate}

\subsection{Structures, the satisfaction relation, satisfiability, validity, the statement of Gödel's completeness theorem}
\begin{enumerate}
    \item Formally define structure
    \begin{itemize}
        \item M
        \item p
        \item f
    \end{itemize}
    \item Give natural interpretation of $(\leq, +, \cdot, 0, 1)$
    \item What do we get with such interpretation?
    \item What is each such interpretation?
    \item How do we interpert a term in a structure?
    \item What is $\sigma$?
    \item What does it map?
    \item Where is a variable mapped?
    \item Where is a function mapped?
    \item How are formulas interpreted?
    \item How do we write an interpretation?
    \item How do we read it?
    \item When are the following interpretations true?
    \begin{itemize}
        \item $m \models_{\sigma} P (t_1, \dots t_n)$
        \item $m \models_{\sigma} t_1 = t_2$
        \item $m \models_{\sigma} \Phi \rightarrow \Psi$
        \item $m \models_{\sigma} \Phi \wedge \Psi$
        \item $m \models_{\sigma} \Phi \vee \Psi$
        \item $m \models_{\sigma} \neg \Phi$
        \item $m \models_{\sigma} \forall x. \Phi$
        \item $m \models_{\sigma} \exists x. \Phi$
    \end{itemize}
    \item What here are atomic formulas?
    \item When is a truth value of $m \models_{\Sigma} \Phi$ independent of $\Sigma$?
    \item How do we write this?
    \item Define a valid sentance
    \item Define a satisfiable sentance
    \item Give examples of valid formulas (2)
    \item Give examples of satisfiable formulas
    \item How are validity and satisfiability conected?
    \item THEOREM - Turing / Church theorem
    \item How do we interpret this theorem?
    \item THEOREM  - Gödels completeness theorem
    \item Give the problem where Gödels theorem is used
    \begin{itemize}
        \item What is the input?
        \item What is the output?
        \item What is this algorithm called?
        \item Is it efficient?
        \item Where is it used?
    \end{itemize}
\end{enumerate}


\subsection{The computational status of validity and satisfiability}
\textit{Lecture 7}
\begin{enumerate}
    \item What algorithm do we get with Gödels completeness theorem and the proof-checking algotrithm?
    \begin{itemize}
        \item Input?
        \item Output?
        \item What is the practical result?
        \item What are the theoretical results? (2)
        \item What algorithm therefore exists? (because of validity)
        \item What is the input?
        \item What is the output?
        \item What practical consequence does it have?
        \item What algorithm therefore exists? (because of satisfiability)
        \item What is the input?
        \item What is the output?
    \end{itemize}
\end{enumerate}

\subsection{Finite satisfiability and bounded satisfiability}
\textit{Lecture 8}
\begin{enumerate}
    \item Create 6 axioms with predicates sever, client, connected
    \item Define a model
    \item What is a model checking question?
    \item What algorithm do we need?
    \item What do we compute/return in the following cases?
    \begin{itemize}
        \item $m \models_{\sigma} P (t_1, \dots t_n)$
        \item $m \models_{\sigma} t_1 = t_2$
        \item $m \models_{\sigma} \Phi \wedge \Psi$
        \item $m \models_{\sigma} \neg \Phi$
        \item $m \models_{\sigma} \exists x. \Phi$
    \end{itemize}
    \item What kind of problem is this?
    \item In what time does algorithm work?
    \item How is this run time called?
    \item Can we make it polynomial?
    \item How?
    \item Depentent on what?
    \item What axioms can we add (in the example from question 1)?
    \item What are we now interested in?
    \item Can you ask this question in a different way? More formally
    \item Algorithm for what do we need?
    \begin{itemize}
        \item Input?
        \item Output?
    \end{itemize}
    \item Is the problem decidable?
    \item What algorithm does exist?
    \item How would a naive algorithm work?
    \item Is it good? Why?
    \item Name the algorithm that is more practical
    \begin{itemize}
        \item What is it's input?
        \item What question does it answer?
        \item What is the output?
    \end{itemize}
\end{enumerate}

\subsection{Reducing bounded satisfiability to SAT solving}
\begin{enumerate}
    \item What is the main idea of bounded satisfiability thorem?
    \item Which stages does the algorithm have?
    \item Describe stage 1
    \begin{itemize}
        \item What is the input?
        \item What do we add?
        \item What do we build?
        \item Using what?
        \item What is the idea?
    \end{itemize}
    \item After we have a sentance in predicate logic, what do we define?
    \item Where does this mapping map...
    \begin{itemize}
        \item Predicate
        \item Negation
        \item Conjinction
        \item Existance
    \end{itemize}
    \item FACT - When is $\Phi$ satisfied with a structure?
    \item How do we get rid of function symbols in the propositional formula?
    \begin{itemize}
        \item With what do we replacee functions?
        \item What axioms do we add? (2)
    \end{itemize}
    \item How do we treat equality =?
    \begin{itemize}
        \item What axioms do we add? (4)
    \end{itemize}
\end{enumerate}

%%%%%%%%%%%%%%%%%%%%%%%%%%%%%%%%%%%%%%%%%%%%%%%%%%%%%%%%%%%%%%%%%%%%%%%%%%%

\section{Linear-time temporal logic}

\subsection{Transition system, LTL syntax, runs and the LTL satisfaction relation}
\textit{Lecture 9}
\begin{enumerate}
    \item Define temporal logic.
    \item Define linear time.
    \item For what is LTL good?
    \item Give an example of something where LTL would be good to use.
    \item What system did we define? Name it.
    \begin{itemize}
        \item How did we represent it? 
        \item What components did we define?
        \item Draw a diagram.
    \end{itemize}
    \item What properties can a LTL system have?
    \begin{itemize}
        \item 1 - name it
        \item 1 - what is the definition of the property?
        \item 1 - does our example system have it? 
        \item 2 - what is the definition of the property?
        \item 2 - does our example system have it?
        \item 2 - give an counterexample.
        \item 3 - name it.
        \item 3 - what is the definition of the property?
        \item 4 - name it 
        \item 4 - what is the definition of the property?
    \end{itemize}
    \item Define a transition system for LTL.
    \begin{itemize}
        \item S
        \item $\rightarrow$
        \item property
        \item L
        \item anything else?
    \end{itemize}
    \item How can we model a deadlock?
    \item What syntax for formulas did we define for temporal logic? (8)
        %   X Goes Forward Until We Rut
    \item Define every symbol.
    \item Define a run.
    \item How do we also call a run?
    \item What notation can we create from a standard notation of a run?
    \item What relation did we define in temporal logic?
    \item What is it's notation?
    \item How do we read the notation?
    \item When is $\Pi \models \dots$ true? (give equality)
    \begin{itemize}
        \item $p$
        \item $\Phi \wedge \Psi$
        \item $\Phi \vee \Psi$
        \item $\Phi \rightarrow \Psi$
        \item $\neg \Phi$
        \item $\bot$
        \item $\top$
        \item $X \Phi$
        \item $G \Phi$
        \item $F \Phi$
        \item $\Phi U \Psi$
        \item $\Phi W \Psi$
        \item $\Phi R \Psi$
    \end{itemize}
\end{enumerate}

\subsection{Examples of fairness constraints and how they are used}
\textit{Lecture 9}
\begin{enumerate}
    \item What is a LTL model checking?
    \begin{itemize}
        \item Input?
        \item Output?
    \end{itemize}
    \item EXAMPLE - What does the algorithm return if we input $G ((t_1 \rightarrow Fc_1) \wedge (t_2 \rightarrow Fc_2))$?
    \item EXAMPLE - Slightly redraw the diagram you drew earlier
    \begin{itemize}
        \item What does the algorithm return?
        \item When would the original model return true?
        \item Define this constraint
        \item With what can we model check?
        \item What does the algorithm return?
    \end{itemize}
\end{enumerate}

\subsection{NBAs (and GNBAs), their associated omega languages and the emptiness condition}
\begin{enumerate}
    \item We defined automats. What are they called?
    \item EXAMPLE - Give a simple example
    \begin{itemize}
        \item Draw a diagram
        \item Give an example of a word this automat would accept.
        \item When does this automat accept a $\omega$-word?
        \item Define a $\omega$-word.
        \item Give some more examples of words that are / are not accepted.
        \item Define $\omega$-language of the automations.
        \item Formally write $\omega$-language.
    \end{itemize}
    \item EXAMPLE - Give another example of a NBA.
    \begin{itemize}
        \item Draw a diagram.
        \item What is $\omega$-language in this example?
    \end{itemize}
    \item EXAMPLE - Give a third example of a NBA.
    \begin{itemize}
        \item Draw a diagram.
        \item What is $\omega$-language in this example?
    \end{itemize}
    \item Define an NBA over an alphabet
    \item When is a $\omega$-word accepted in an NBA over an alphabet?
    \item PROPOSITION - When is a $\omega$-language of NBA non empty?
    \item What are we testing here?
    \item Define $\omega$-regular 
    \item THEOREM - How can we construct new $\omega$-regular sets? (4)
    \item Define a general NBA.
    \item When is a word accepted?
    \item EXAMPLE - of a GNBA.
    \begin{itemize}
        \item Draw a diagram.
        \item What is the alphabet?
        \item What is the $\omega$-language?
    \end{itemize}
    \item PROPOSITION - How are NBA and GNBA connected? \textit{ucilnica}
    \item THEOREM - What holds for all GNBA?
\end{enumerate}

\subsection{Trace language of an LTL formula}
\textit{Ucilnica lecture 10}
\begin{enumerate}
    \item What do we have to consider?
    \item What holds for the systems we define?
    \item What do we have to suppose?
    \item What follows for all LTL formulas?
    \item On what does $\pi \models \Phi$ depend?
    \item What do we therefore define?
    \item How can we also look at the trace?
    \item What does a formula $\Phi$ determine?
    \begin{itemize}
        \item What is it called?
        \item How do we "notate it"?
    \end{itemize}
    \item Formally write out $Trace(\Phi)$.
    \item How do we connect GNBA and Trace?
    \item What assumption do we make on $\Phi$?
    \item How can we express $F \phi$ with these connectives? \textit{zapiski}
    \item How can we express $G \phi$ with these connectives? \textit{zapiski}
    \item Define a completed subformula of $\Phi$. (6 cases)
    \item How is $CS(a U b) $ defined?
    \item Define an elementary subset. (5 cases)
    \item EXAMPLE - What are elementary subsets of $CS(a U b)$? (5 cases)
    \item Draw a diagram
    \item What do we have for every $\Phi U \Phi'$.
    \item Define $S \rightarrow^{A} S'$. (5 cases)
    \item THEOREM - What is a subset of $A_{\Phi}$.
\end{enumerate}

\subsection{The model checking problem for LTL}
\textit{Ucilnica lecture 10}
\begin{enumerate}
    \item What is the input?
    \item What is the question?
    \item STEP 1
    \begin{itemize}
        \item What do we construct?
        \item To what is it converted?
        \item To what does an accepted run in converted structure correspond?
    \end{itemize}
    \item STEP 2
    \begin{itemize}
        \item What do we construct?
        \item What are the accepted runs in this structure?
    \end{itemize}
    \item How do we construct a product NBA?
    \begin{itemize}
        \item Define $m$
        \item Define $A$
        \item Define a set of states
    \end{itemize}
    \item How do we get an NBA from GNBA? 
    \item STEP 3
    \begin{itemize}
        \item What do we test?
        \item How do we test this?
        \item When do we output YES?
        \item When do we output NO?
    \end{itemize}
    \item EXAMPLE - Illustrate the model checking on $m, s_0 \models \neg (a U b)$
\end{enumerate}

%%%%%%%%%%%%%%%%%%%%%%%%%%%%%%%%%%%%%%%%%%%%%%%%%%%%%%%%%%%%%%%%%%%%%%%%%%%

\section{Hoare logic (logic for verifing programs)}
\textit{Lecture 11 + ucilnica}

\subsection{Proof rules for partial correctness}
\begin{enumerate}
    \item Where is it used?
    \item Why is hoare logic important?
    \item In what syles can we write proof?
    \item Give proof rules in tree style
    \begin{itemize}
        \item Partial while
        \item Composition 
        \item Assignment
        \item Skip
        \item Conditional
        \item Consequence 
        \item What is the side condition for the consequence rule?
    \end{itemize}
    \item Give tableaux rules
    \begin{itemize}
        \item Assignment
        \item Skip
        \item Partial while 
        \item What is the precondition for while?
        \item Implied
        \item If statement
        \item What are the preconditions for if?
    \end{itemize}
\end{enumerate}

\subsection{Statement of soundness}
\begin{enumerate}
    \item Define $\models_{par} \{ \phi \} C \{ \psi \}$
    \item Define $\vdash_{par} \{ \phi \} C \{ \psi \}$
    \item Soundness theorem
    \item How is the soundness theorem proved?
    \item LEMMA - Proving preservation property of partial while
    \item Proof
    \begin{itemize}
        \item How do we proof this?
        \item What has to be satisfied?
        \item Write proof for $n=0$
        \item Write proof for $n - 1 \rightarrow n$
    \end{itemize}
\end{enumerate}

\subsection{Total correctness}
\begin{enumerate}
    \item Proof rule for total while in tree style
    \begin{itemize}
        \item What is $z_0$?
        \item What is $\eta$?
        \item What is $E$?
    \end{itemize}
    \item Proof rule of total while in tableaux style
    \item Define $\models_{tot} \{ \phi \} C \{ \psi \}$
    \item Define $\vdash_{tot} \{ \phi \} C \{ \psi \}$
    \item Soundness theorem
    \item LEMMA - Proving preservation property of partial while
    \item Proof
    \begin{itemize}
        \item How do we proof this?
        \item What has to be satisfied?
        \item Write proof for $n=0$
        \item Write proof for $n - 1 \rightarrow n$
    \end{itemize}
\end{enumerate}

\subsection{Completeness for partial correctness}
\begin{enumerate}
    \item Completeness theorem
    \item How is completeness theorem also called?
    \item Why is it called so?
    \item What is Gödels theorem called?
    \item Define notation $wp(C, \psi)$
    \item Name 3 lemmas that use $wp$
    \item LEMMA - expressive completeness
    \item LEMMA - Weakest precondition
    \item LEMMA - Sequencing
\end{enumerate}

\subsection{Proof of completeness}
\begin{enumerate}
    \item How is the proof structured?
    \item What case of the proof did we show?
    \item What induction hypothesis do we have for this case?
    \item What did we suppose?
    \item What do we have to proove?
    \begin{itemize}
        \item 1
        \item 2
        \item 3
    \end{itemize}
    \item What follows if these 
    \item Prove 1
    \item Prove 2
    \item Prove 3
\end{enumerate}


\end{document}